% Using "%" you can add comments that will not be displayed in the finished document
%
% Declares the document class
% Available formats: article, proc, minimal, report, book, slides, memoir, letter and beamer. Other types must have a .cls file included
\documentclass{article}

% Include libraries using "\usepackage{packageName}"

% Document beginning
\begin{document}

% ========== TITLE ==========
% Title
\title{Hello World!}
% Author and Date
\author{Pacman}
% Generate title
\maketitle
% ========== ===== ==========

% New section title
\section{Exercise 1}

Some text.....\\
\textbf{This is bold} ($\backslash$textbf\{some text\})\\
 \textit{This is italic} ($\backslash$textit\{some text\})\\
 \underline{This is underlined} ($\backslash$underline\{some text\}) \\ \\ % <- Double backwards slash for new line
To begin a new line, use double backward slash $\backslash\backslash$ \\
To skip a page, use $\backslash$newpage or $\backslash$pagebreak \\
To use mathmode, use the dollar sign on either sign (\$) of an equation like:\\
\$x/y=10\$ $\rightarrow$ $x/y=10$ \\
$f_{x} = ax^{10^x} + a^2x * b_{g_z}$\\
$\bullet$ To find more mathematical symbols, go to Detexify (http://detexify.kirelabs.org/classify.html) \\

You can create tables using the LaTeX table generator (https://www.tablesgenerator.com/) like:\\

\begin{table}[h!]
\centering
\begin{tabular}{|c|c|c|c|c|}
\hline
\textit{\textbf{a}} & \textit{\textbf{b}} & \textit{\textbf{c}} & \textit{\textbf{d}} & \textit{\textbf{e}} \\ \hline
\textit{\textbf{1}} & \textit{\textbf{2}} & \textit{\textbf{3}} & \textit{\textbf{4}} & \textit{\textbf{5}} \\ \hline
\textit{\textbf{6}} & \textit{\textbf{7}} & \textit{\textbf{8}} & \textit{\textbf{9}} & \textit{\textbf{0}} \\ \hline
\textit{\textbf{1}} & \textit{\textbf{2}} & \textit{\textbf{3}} & \textit{\textbf{4}} & \textit{\textbf{5}} \\ \hline
\end{tabular}
\end{table}
Make sure to add the h! to the brackets in the generated table to force it to be put where you want, otherwise it will float around freely.\\ \\

Using $\backslash$centering you can type text in the middle like:
\begin{figure}[h!]
\centering
Some text...
\end{figure}

Using $\backslash$includegraphics you can add a picture:\\ \\
$\backslash$begin\{figure\}$[$h!$]$\\
$\backslash$includegraphics $[$width=linewidth$]$\{fileName.jpg/png\}\\
$\backslash$end\{figure\}

% Document ending
\end{document}